\documentclass{article}
\usepackage{boxedminipage}
\usepackage{url}
\usepackage[usenames]{color}
\usepackage[normalem]{ulem}

\definecolor{LtGray}{rgb}{0.9,0.9,0.9}

\newcommand{\ModelE}{{\tt ModelE}}
\newcommand{\HlxIcons}{}
\newcommand{\require}[1]{{\noindent
    \fbox{{\colorbox{LtGray}{\parbox[t]{4.8in}{\noindent {\scriptsize \color{Red} {\textsc{\bf Mandatory}}} \newline #1}
}}}}}

\newcommand{\recommend}[1]{{{\noindent
    \fbox{\colorbox{LtGray}{\parbox[t]{4.8in}{\noindent {\scriptsize \color{Blue} {\textsc{\bf Encouraged}}} \newline #1}
}}}}}

\title{\ModelE~Coding Conventions}
\author{Tom\ Clune \\
NASA Goddard Space Flight Center \\
email: \emph{thomas.l.clune@nasa.gov}
}
\date{\today}
\begin{document}
\maketitle
\tableofcontents
\section {Introduction}

This document establishes certain common coding conventions within
\ModelE~software.  With the overarching goal of improving science
productivity, these coding conventions are intended to 
\begin{itemize}
\item reduce common causes of bugs and/or inscrutible software,
\item improve overall software quality,
\item reduce differences in coding styles that limit legibility, and
\item enable the use of automatic sofware development tools (e.g. Photran).
\end{itemize}

Documents analogous to this are an increasingly common practice among
commercial software development organizations and are widely believed
to improve productivity through a number of direct and indirect
impacts.  No doubt the balance of these drivers are somewhat different
in research organizations, and the set of conventions below are
intended to be a compromise among conflicting ideals of best
practices, existing coding conventions, and other unique requirements
of \ModelE.  Where possible, each requirement and/or recommendation is
provided with rationale in the hopes of providing a compelling
motivation.

The conventions in this document will be periodically reviewed,
updated, and extended to ensure maximum benefit.

\subsection {Mandatory vs. Voluntary}

For the most part the establishment of these conventions these
conventions is \emph{not intended to be disruptive} to ongoing work, but
rather to guide a gradual transformation as the community becomes more
comfortable with the various elements.  In that spirit, please note
that \emph{most} conventions in this document are \emph{not}
considered to be mandatory for \ModelE~developers.  Developers who find
themselves uncomfortable with any items should continue with their
existing coding style and/or discuss their concerns with any member of
the core software engineering team.   

\section {Naming conventions}

Named software entities (variables, procedures, etc) are perhaps the
most important mechanism by which one communicates and understands the
intent an implementation.  The choice of a good name can be
challenging in many instances, but is also often a rewarding
opportunity for creativity.

\subsection {General Guidelines}

Explicit absolute naming rules would be difficult to produce and most
likely counter-productive in practice.  Instead \ModelE~developers
should focus on general principles for good names and use their own
experience and judgment in the final selection.   

The importance of name selection generally increases with the scope
for an entity.  Thus, names for input parameters are the most critical
followed by names for public module variables, subroutines, and
functions.  Next lower in priority would be names of dummy
arguments. And lowest in priority would be names of local variables
and private module variables. Names that are used more frequently are
worth greater investment than names that are used very infrequently.

\subsubsection {Communicate intention}
Good names should communicate the intention of a given software entity
unambiguously to other developers.  The name should give a useful
indication of the role that the entity serves in the software using
terminology that is understandable by other developers.  Generally
this guideline implies a preference for full English words and
phrases, with the understanding that numerous caveats and exceptions
exist.  As an example consider the choice for naming a variable which
contains the heat flux at the bottom of a grid-cell.  The variable
names ``f'' or ``Q'' are common in this situation, but within a large
routine do not generally provide much insight to other developers. The
name ``flux'' is better, but still lacks a certain degree of
specificity.  The name ``heatFlux'' or ``heatFluxQ'' or ``lowerFlux''
are better and depending on context might be sufficient.  In a large
routine with multiple types of fluxes at various boundaries, a better
name would be ``lowerHeatFlux'' or ``heatFluxAtBottom''.   

Perhaps the worst offense against the guideline here would be reusing
a variable with a perfectly fine name for a second very different
purpose.  E.g. reusing ``heatFlux'' at a later point to represent
something like the total mass.


\subsubsection {Consistency, predictability, and ambiguity}
Developers should not need to unnecessarily spend time determining the
correct spelling of a given name.  Abbreviations of long words are
perfectly natural so long as they are \emph{consistently applied
  and predictable}.  When multiple abbreviations or alternate
spellings (or even misspellings!) are in common use, developers
\emph{must} frequently check other pieces of code to ensure they are
using the correct spelling.  For example the name ``trop'' is probably
poor for indicating a tropospheric quantity if ``tropo'' and
``troposph'' are also used.  At some point in the future, a table of
abbreviations will be added to this document.

\recommend{Avoid abbreviations in names unless consistent throughout \ModelE.}

\recommend{Use correct spelling. Where English and American
  spellings differ, the American spelling should be preferred.}

\subsubsection {Generic terminology}
Although very tempting, certain common words are too generic to confer
any useful information to other developers and are generally poor
candidates for parts of a variable name.  Examples of such bland terms
include:

\noindent
\begin{tabular}[t]{ll}
\parbox[t]{2in}{
\begin{itemize}
\item variable
\item parameter
\item buffer
\end{itemize}} &
\parbox[t]{2in}{
\begin{itemize}
\item string
\item array
\item table
\end{itemize}}
\end{tabular}

\noindent
When tempted to use such terms in a name, consider other aspects of
the functionality to come up with alternatives.

\recommend{Avoid bland or generic terms in names of variables.}

\subsubsection {Name length}
The guidelines given above generally drive selection toward longer
names which convey more information.  Clearly there are advantages to
shorter names as well, and a good compromise is a bit of an art form.
Note that concern about time spent (wasted) typing longer names is
generally misplaced, as numerous studies have shown that source code
is read many more times than than it is written.  Further, many
modern software editors provide means to ``auto-complete'' names,
which further reduces concerns over typing long names.

Names which are \emph{too} long can reduce clarity, especially
in long expressions.  When the discrepancy is severe, there are
several alternatives:
\begin{enumerate}
\item Split the long expression into multiple statements by
  introducing intermediate variables for subexpressions.  This often
  improves the clarity in a number of ways with the intermediate names
  providing new avenues for communication.
\item Introduce a local variable with a shorter name to be used as an
  alias.  Because the new name has a smaller scope and is directly
  associated with the original variable, a very short string is very sensible.
\item In the near future the F2003 {\tt associate} construct will
  provide a formal mechanism for using a short name (alias) to
  represent repeated subexpressions within a longer expression.
\end{enumerate}

\require{In no event shall a name exceed 31 characters which is the maximum
under the F2003 standard.}
The F2008 standard will extend this to 63 characters, but this is
motivated by the need to support automatically generated source code,
and should \emph{not} be seen as guidance for human-generated software.

\subsection {Specific conventions}

\subsubsection {Multi-word Names}

\recommend{\ModelE~will use the common so-called mixed-case convention
  for concatenating multiple words in a variable name.}

\noindent In this convention the beginnings of words are indicated by
starting them with capital letters, e.g. ``potentialTemperature'' and
``numTracers''.  Capitalization of the first word is context dependent
and discussed in more detail below.  Although this convention is
somewhat arbitrary, many groups have adjusted to this convention and
grow to prefer it.  It is important that a single convention be
established as it eliminates time spent determining whether a given
variable uses some other mechanism to append words.  Also, although
Fortran is case-insensitive, consistent capitalization aids in reading
code and finding other instances of the same variable.  (Not to
mention simply eliminating debate about which capitalization to use in
the first place.)


\subsubsection {File names}

As with variable names, file names should communicate their intent
which should be their contents.  In this sense, files should ideally
contain only one entity which will either be a program, a subroutine,
a function or a module.  The current implementation of \ModelE~is far
from this ideal, and adoption is expected to be very gradual.

\recommend{Choose file names to coincide with its contents.}

\noindent The suffix of a file name is to be used to indicate whether the overall
format is \emph{fixed} or \emph{free}.

\require{Fixed format files \emph{must} end
with the {\tt .f} or {\tt .F} suffix, while free format files \emph{must} end
with {\tt .F90}.}

\noindent For example, given a software entity named {\tt foo}, the
corresponding free-format file name should be {\tt foo.F90}.

\subsubsection {Derived type names}
Derived type names should end with the {\tt \_type} suffix to indicate
their role.  This convention might change once F2003 becomes more
widespread and other object-orient conventions will be more
appropriate.  Fortran 95 did not permit module procedures to have the
same name as derived types which is a natural situation for
constructor methods.  F2003 relaxes this restriction.

\recommend{Use the {\tt \_type} suffix for names of Fortran derived types.}

In analogy with object-oriented languages where developers typically
capitalize class names, derived type names should be capitalized.  The
issue is less important in Fortran since the {\tt type} keyword is
always present for derived types.


\subsubsection {Module names}

Modules are sufficiently fundamental that reserving a special suffix
to indicate their names is a sensible and common convention.  Most
communities have opted to use {\tt Mod} suffix for this purpose.  This
is also the recommendation for \ModelE, but with special exemptions
related to existing conventions for physical components within the
model.  Files containing a module should also follow the convention of
dropping the {\tt Mod} suffix in the file name.  In that context the
suffix is somewhat redundant, and dropping the suffix is more
consistent with the style of other community software.  As with
derived type names, it is generally appropriate to capitalize module
names.


\recommend{Most module names should use the {\tt Mod} suffix.}

\recommend{The {\tt Mod} suffix should be omitted from the name of a
  file containing a module.}

\recommend{Capitalize module names.}

\paragraph {Subsystem global entities module \_COM}

A consistent existing convention within \ModelE~is for modules which
provide the various global variables associated with a given physical
component.  The modules are currently named with the {\tt \_COM}
suffix, and warrant an exception from the usual naming convention for
modules.  In most instances this convention is already consistent with
the corresponding file name, but will eventually require a fix for th exceptions.

\paragraph {Subsystem driver module \_DRV}

In \ModelE~a consistent existing convention for most physical
components is to have a top level file containing the suffix {\tt
  \_DRV}.  This convention is also to be continued, but the
corresponding procedure names are generally quite inconsistent with
this convention.  E.g. the file {\tt RAD\_DRV.f} contains the top-level
procedure {\tt RADIA()}

Both of the preceeding two exceptions are likely to be revisited if
and when these physical components are re-implemented as ESMF
components.


\subsubsection {Procedure names}
Subroutines and functions perform actions and are generally best
expressed with names corresponding to English verbs.  E.g. {\tt
  print()} or {\tt accumulate()}.  Many routines are intended to put
or retrieve information from some sort of data structure, possibly
indirectly.  The words ``put'' and ``get'' are useful modifiers in
such instances.  E.g. {\tt putLatitude} or {\tt getSurfaceAlbedo()}.
Although these conventions are fairly natural, actual awareness of
them of can be beneficial when creating names.

\subsubsection {Variable names}
Variable names represent objects and as such are generally best
represented with names corresponding to English nouns.  A good
rule-of-thumb is to use singular nouns for scalars and plurals for
lists/arrays.  Note, however, that this rule-of-thumb has a very
important exception for arrays which represent spatially distributed
quantities such as {\tt temperature(i,j,k)} which are referred to in
the singular by common convention.

\section {Fortran language constructs}

\subsection {Which Fortran version}

In an ideal world, \ModelE~would to be implemented in strict
compliance with the Fortran standard.  However, allowance \emph{must} be
given to the evolution of the Fortran standard itself as well as to a
very small number of nonstandard, yet highly portable extension to the
Fortran language.  At the time of this writing (January 2010), the
current standard is Fortran 2003 (F2003) and the Fortran 2008 (F2008)
standard is expected to be fully ratified later this year.  In
reality, few Fortran compilers have implemented the full F2003
standard and the interests of \ModelE~portability require that source
code be restricted to a more portable subset of F2003 defined as that
which is supported by current version of both GFortran \emph{and}
Intel Fortran compilers.  \ModelE~execution under GFortran guarantees
a strong degree of portability, while Intel guarantees continuity and
high performance for GISS's primary computing environments.  Note that
some other compilers most likely also support this subset of F2003 (and
beyond), so this constraint is not as severe as it might first appear.

\require{\ModelE~is implemented in the subset of Fortran 2003 that is
  robustly implemented by both current Intel and GFortran compilers.}


\subsubsection {Non standard extensions in \ModelE}
\begin{description}
\item [CPP] The build process of \ModelE~relies upon the C
  preprocessor (CPP), which is technically not part of the Fortran
  standard.  This capability is essential for enabling multiple
  configurations of the model.
\item [real*8] Although the Fortran 90 standard introduced portable
  syntax for controlling the precision of floating point quantities,
  the widespread extension ({\tt real*8, real*4}) is portable on
  virtually all Fortran compilers and deeply embedded in \ModelE.  The
  Fortran {\tt KIND=} mechanism is of course permitted and encouraged
  in software sections where support of multiple precisions is
  required.
\end{description}

\subsection {Obsolete and discouraged features}

Due to the desire to support legacy software, the Fortran standard
rarely actually removes language features even when superior
mechanisms have been introduced.  \ModelE~developers are strongly
encouraged to avoid the following language features:

\paragraph{\bf {\tt entry} statement} At best this mechanism has
always been confusing, and far better mechanisms now exist to share
functionality across multiple interfaces.  This feature is strictly
forbidden from being added to \ModelE, and all existing uses will soon
be eliminated.   This change is further motivated by some software tools
which do not support this language ``feature''.

\require{The {\tt entry} statement should not be used in \ModelE.}

 \paragraph{\bf{arithmetic {\tt if}}} Although compact, this construct
 generally obfuscates code.

\require{The arithmetic {\tt if} construct should not be used in \ModelE.}

 \paragraph{\bf {computed {\tt goto}}} This feature is generally
   inferior to the newer {\tt select case} construct which shows the
   conditions for execution at the top of each case. 

\require{The computed {\tt goto} construct should not be used in \ModelE.}

   \paragraph{\bf {\tt goto} statement} Although there are still
   certain situations where the use of {\tt goto} is the clearest
   expression of an algorithm, such situations are vanishingly rare in
   practice.  The {\tt cycle} and {\tt exit} statements generally
   communicate intent in a superior manner within loops, and {\tt
     select case} and plain old {\tt if} statements cover most other
   cases.

\recommend{Alternatives to the {\tt goto} statement should
     be be used.}

\paragraph{\bf {\tt continue} statement} {\tt END DO} is generally the
  preferred mechanism to close loops.  For longer loops where the loss
  of a statement label might complicate finding the corresponding
  beginning of a loop, developers should use the F90 mechanism for
  labeling blocks. E.g.
\begin{verbatim}
outerLoop: do i = 1, 10
...
end do outerLoop
\end{verbatim}
\recommend{Avoid the use of the {\tt continue} statement.}

  \paragraph{{\bf statement labels}} Although these are still
  necessary for {\tt goto} statements which cannot yet be removed, other
  uses should rely on the F90 mechanism for labeling blocks.

\recommend{Use F90 statement labels for long nested loops that
    extend more than one screen.}


\subsection {Required and encouraged features}

Accidental misspelling of variables was once a common source of errors
in Fortran programs.  The introduction of {\tt implicit none} has
alleviated many such errors and fortunately has become widely used.

\require{The {\tt implicit none} statement \emph{must} be used in all
  modules and all non-module subroutines and functions.}

By default all Fortran module entities are ``public'' which can lead
to problems with multiple paths by which those entities are accessed
by higher level program units.  The cascade of possible host
association can lead to long and/or aborted compilation.  Aside from
these technical issues, one of the intents of the Fortran module
construct is to encapsulate (i.e. hide) details of implementation from
external program units.  Fortunately, Fortran has the {\tt private}
statement which toggles this default.

\recommend{Modules should use the {\tt private} statement.  Entities
  which should be accessible by other program units should be declared
  with the {\tt public} attribute.}

Even more than Fortran modules, derived types should hide the details
of their internal implementation.  Unfortunately, as with modules, the
default public access leads to over-reliance on access to internal
details. With F95 such structures must be entirely public or entirely
private, but F2003 introduces finer control.

\recommend{Fortran derived types should use the {\tt private}
  statement where possible.}


\section {Formatting conventions}

Formatting issues are far less substantive than the software elements
that are discussed earlier in this document.  However, a consistent
``look-and-feel'' can be a powerful aid to the readability of \ModelE
as well as preventing needless thrashing in CVS as one developer after
another imposes their personal preference.  Nonetheless, this section
is intentionally minimalist and as much as possible reflects
existing style within \ModelE.

\subsection {Free format}

Although \ModelE is at the time of this writing almost exclusively
implemented in the older fixed-format Fortran convention, the new
default format is exclusively free-format.  Further, the existing code
base will soon be thoroughly converted to free-format.  While there
are several minor advantages to free-format, the rationale for the
wholesale conversion is to leverage a new generation of powerful
software tools that do not support the older format. 

Although, free-format permits source code to extend up to column 132,
practical readability requires that source code be limited to column
80.  Exceptional cases where the code marginally exceeds this
threshold may be acceptable if additional line-splits have comparable
consequences on appearance.

\subsection {Indentation}

The interior of each of the following categories of Fortran code
blocks shall be indented in a consistent manner:

{\tt
\begin{tabular}[t]{|l|l|}
\hline
\parbox[t]{2in}{module\\ \hspace*{0.15in}<indented block>\\end module\\ } &
\parbox[t]{2in}{subroutine\\ \hspace*{0.15in}<indented block>\\end subroutine\\} \\
\hline
\parbox[t]{2in}{function\\ \hspace*{0.15in}<indented block>\\end function\\} &
\parbox[t]{2in}{program\\ \hspace*{0.15in}<indented block>\\end program\\} \\
\hline
\parbox[t]{2in}{type\\ \hspace*{0.15in}<indented block>\\end type\\} &
\parbox[t]{2in}{interface\\ \hspace*{0.15in}<indented block>\\end interface\\} \\
\hline
\parbox[t]{2in}{if (...) then\\ \hspace*{0.15in}<indented block>\\else\\ \hspace*{0.15in}<indented block>\\ endif\\} &
\parbox[t]{2in}{select case \\ case (...)\\ \hspace*{0.15in}<indented block>
\\ case (...)\\ \hspace*{0.15in}<indented block>\\end select\\} \\
\hline
\parbox[t]{2in}{do\\ \hspace*{0.15in}<indented block>\\end do\\} & \\
\hline
\end{tabular}
}

\vspace*{0.1in}
At this time precisely 2 spaces shall be used for each level of
indentation.  Although a larger indentation is generally preferable
for readability, existing reliance on very deep nesting is a dominant
concern.  If at some later time, deep nests have been eliminated from
\ModelE, the level of indentation will be raised.

Indentation should always be implemented with spaces, as the {\tt
  <TAB>} character is not legal in Fortran source code.
Unfortunately, some common editors will permit the insertion of {\tt
  <TAB>} characters, so some caution is appropriate.  Note to Emacs
users: Although the {\tt <TAB>} key is used to auto-indent lines of
source code in Fortran mode, the editor actually only inserts (or
removes) spaces to achieve indentation.

\subsubsection {Indentation of documentation}
Documentation in the header of procedures and modules should not be
indented, while documentation lines in executable sections should be
indented at the same level as the surrounding code.  End-of-line not
extend beyond column 80.

\subsection {Spacing}
\subsubsection {Two word Fortran keywords}
Although spaces are generally significant under the free-format
convention, for most (possibly all?) compound keywords (e.g. {\tt end
  do} and {\tt go to}) the intervening space is optional.  For \ModelE~
the convention is to require the intervening space for all such constructs except for {\tt goto}:

\noindent
{\tt
\begin{tabular}[t]{ll}
\parbox[t]{2in}{
\begin{itemize}
\item goto
\item end do
\item end if
\item end select
\item end subroutine
\item end function
\item end subroutine
\end{itemize}} &
\parbox[t]{2in}{
\begin{itemize}
\item \sout{go to}
\item \sout{enddo}
\item \sout{endif}
\item \sout{endselect}
\item \sout{endsubroutine}
\item \sout{endfunction}
\item \sout{endsubroutine}
\end{itemize}}
\end{tabular}
}

\require{Use a space between compound keywords except for the {\tt goto} statement.}

\subsubsection {Operators}

\recommend{To improve legibility, expressions should attempt to use the space
character in a judicious manner.}  The rules here are not absolute, but
guidelines that should be followed unless other legibility issues are
more important.  In order of decreasing priority one should:
\begin{itemize}
\item Use at least one space should be left on each side of the
assignment (``{\tt =}'') operator.
\item Use at least one space on each side of ``+'' and ``-'' operators
  to both emphasize grouping as well as order of precedence among operators.
\item \emph{Not} use space around ``*'' and ``**'' operators.
\item Use one space after ``,'' in arguments to procedures and functions.
\item \emph{Not} use space between array indices.
\end{itemize}

\subsection{Capitalization}

Although Fortran is case insensitive, capitalization can be useful to
convey additional information to readers.  Because modern editors can
generally highlight language keywords, capitalization is generally
only to be applied to user-defined entities.  As mentioned above,
capitalization should be used to separate words within multi-word
names, as well as for derived type and module names.

\recommend{Use lower case for Fortran keywords.}
\recommend{Use mixed case for multiword names.}
\recommend{Start names with lower case except for derived types and modules.}

\section {Documentation}

\ModelE~uses scripts to dynamically assemble certain documentation
from source code in an automated manner based upon special
identification tags.

\subsection {Documentation of Fortran modules}

Each module \emph{must} have a top-level summary indicated with the comment
tag: ``{\tt !@sum}''.  This summary should explain the nature of the
modules contents and the role of the module within the context of the
overall model.

All global (i.e. {\tt public} module entities \emph{must} be documented with
the comment tag: ``{\tt !@var}''.  This documentation should emphasize
the purpose of the entity, and for physical quantities the
documentation should specify the physical units (e.g. ``{\tt m/s}'').

Where appropriate each module should specify the primary author(s) or
point(s)-of-contact with the comment tag: ``{\tt !@auth}''.  For more
complex situations, the repository is a better mechanism for
determining which developers are responsible for any bit of code.


\subsection {Documentation of Fortran procedures}

\require{Each public procedure (subroutine or function) \emph{must} have a top-level
summary indicated with the comment tag: ``{\tt !@sum}''.}  This summary
should explain the nature of the modules contents and the role of the
module within the context of the overall model.

\require{Each procedure dummy variable \emph{must} be documented with the comment tag:
``{\tt !@var}''.}  This documentation should emphasize the purpose of
the entity, and for physical quantities the documentation should
specify the physical units (e.g. ``{\tt m/s}'').

\recommend{Important/nontrivial local variables should be also be documented with
the ``{\tt !@var}'' tag.}

\recommend{Where appropriate and/or different than for the surrounding module,
each procedure should specify the primary author or point-of-contact
with the comment tag: ``{\tt !@auth}''.}  For more complex situations,
the repository is a better mechanism for determining which developers
are responsible for any bit of code.

\subsection {Documentation of rundeck parameters}

Rundeck parameters are among the most important quantities from the
point-of-view of other users of the software, and strong documentation
for those parameters is a very high priority.

\require{All rundeck parameters \emph{must} be documented using the comment tag
  ``{\tt !@dbparam}''.}

\section {Miscellaneous}

\subsection {Free format templates}

Some users may find it convenient to begin new modules and/or
procedures with a skeleton implementation that indicates such things
as proper indentation and other conventions.
Figure~\ref{module-template} provides a template for Fortran modules
that conforms to the conventions established in this document.
Figure~\ref{subroutine-template} provides an analogous template for
Fortran subroutines, and figure~\ref{function-template} provides a
template for Fortran functions.


\begin{figure}[h]
\begin{boxedminipage}[t]{4.8in}
\begin{verbatim}
module <module-name>Mod
!@sum <summary>
!@auth <principle author>
  use <use-module>, only: <item>
  ...
  implicit none
  private

  ! list public entities
  public :: <item>
  ...

  ! declare any public derived types
  type <name>_type
    private
    <declare components of derived type>
  end type <name>_type
  ...

  ! declare public variables
  !@var <var1> <description> <units>
  real*8, allocatable :: <var1>(:,:,:)
  ...

contains

  <procedure 1>

  <procedure 2>

end module <module-name>Mod
\end{verbatim}
\end{boxedminipage}
\caption{Template for Fortran module in free-format.}
\label{module-template}
\end{figure}

\begin{figure}[h]
\begin{boxedminipage}[t]{4.8in}
\begin{verbatim}
subroutine <routine-name>(<arg1>[, <arg2>, ...])
!@sum <summary>
!@auth <principle author>
  use <use-module>, only: <item>
  ...
  implicit none ! not required for module subroutine

  ! declare dummy arguments
  !@var <arg1> <description> <units>
  real*8, allocatable, intent(...) :: <arg1>(:,:)

  ! declare local variables
  real*8, allocatable :: <var1>(:,:)


  <executable statement>
  ...

end subroutine <routine-name>
\end{verbatim}
\end{boxedminipage}
\caption{Template for Fortran subroutine in free-format.}
\label{subroutine-template}
\end{figure}


\begin{figure}[h]
\begin{boxedminipage}[t]{4.8in}
\begin{verbatim}
function <routine-name>(<arg1>[, <arg2>, ...])
!@sum <summary>
!@auth <principle author>
  use <use-module>, only: <item>
  ...
  implicit none ! not required for module subroutine

  ! declare dummy arguments
  !@var <arg1> <description> <units>
  real*8, allocatable, intent(...) :: <arg1>(:,:)

  ! declare return type
  real*8 :: <routine-name>

  ! declare local variables
  real*8, allocatable :: <var1>(:,:)


  <executable statement>
  ...

end function <routine-name>
\end{verbatim}
\end{boxedminipage}
\caption{Template for Fortran function in free-format.}
\label{function-template}
\end{figure}

\clearpage

\subsection {Emacs settings}

The Emacs editor has a number of useful features for editing
free-format Fortran files.  However, the default settings
(e.g. indentation) do not correspond to the conventions established in
this document.  The elisp code in figure~\ref{elisp}, when inserted
into a users {\tt .emacs} file, will cause Emacs to automatically
recognize files ending in ``{\tt .F90}'' or ``{\tt .f90}'' as
free-format and set the default indentation to be 2 characters.

\begin{figure}[h]
\begin{boxedminipage}[t]{4.8in}
\begin{verbatim}
; Ensure that F90 is the default mode for F90 files
(setq auto-mode-alist (append auto-mode-alist 
                        (list '("\\.f90$" . f90-mode) 
                              '("\\.F90$" . f90-mode))))
; ModelE F90 indentation rules
(setq  f90-directive-comment-re "!@")
(setq  f90-do-indent 2) 
(setq  f90-if-indent 2) 
(setq  f90-program-indent 2)
(setq  f90-type-indent 2)) 
(setq  fortran-do-indent 2)
(setq  fortran-if-indent 2)
(setq  fortran-structure-indent 2)
\end{verbatim}
\end{boxedminipage}
\caption{Elisp code to customize Emacs environment for \ModelE~conventions.}
\label{elisp}
\end{figure}


\end{document}

